\documentclass[11pt, a4paper]{report}

\usepackage [francais] {babel}		% dire que l'on veut rédiger en français
\usepackage[utf8]{inputenc}		% dire que l'on veut utiliser tous les caractères du clavier
\usepackage[T1]{fontenc}		% dire que l'on veut utiliser tous les caractères du clavier
\usepackage{graphicx}
\usepackage[urlcolor=blue,colorlinks=true,linkcolor=blue]{hyperref}			% inclure pdf, png...
\usepackage{pdfpages}

\title{Projet réseau S5 L3-I}
\author{Genevois / Thiercé / Ambrois / Chaumienne / Breton / Mahé}
\date{mars 2014}

\begin{document}
\maketitle
\renewcommand{\contentsname}{Sommaire}
\tableofcontents

\part{Introduction}

  \chapter{Contexte de réalisation}
    
    Ce projet a été réalisé dans le cadre du module réseau de L3 informatique de l'université de Cergy Pontoise. Il devait etreréalisé
    par une équipe de six personnes dont certaines avec des rôles précis :
    \newline
    
    \begin{itemize}
    
     \item Animateur et chef de projet : Thibaut Genevois
     \item Architecte technique : Nicolas Thiercé
     \item Secrétaire : Guillaume Ambrois
     \item Gardien du temps : Kévin Breton
     \item Scribe : Charles Chaumienne
     
    \end{itemize}
    
  \chapter{Sujet}
    
    L'objectif était de concevoir un jeu de type MMO basé sur une architecture client serveur. Le programme client devra
    être écrit en Java et le serveur en C.
    
\part{Spécifications}

  \chapter{Spécifications du jeu}
    
    authentification, se balader
    
  \chapter{Spécifications de la couche réseau}
  
    \section{Protocole réseau utilisé}
  
    \section{Protocole applicatif}
    
    \section{Les paquets}
    
    \section{Gestion des erreurs}
    
\part{Implémentation}
  
  \chapter{Client}

	\section{GUI}

	Avant de parler de la constitution de l'interface graphique, je vous propose de visualiser le résultat. La première figure vous dévoile la page de connexion. Un formulaire de connexion on ne peut plus simple mais entièrement fonctionnel.
	La seconde, vous montre le jeux avec deux personnage de connecté. L'un est le notre, devinerez vous lequel?
	
	
	>>>>>>>>>>>>>>>>>>>>>>>>>>>>>>>>>>>>>>>>>>
	
	FIGURES A INSERER 
	figure 1 : Ecran de connexion
	figure 2 : Deux personnages en jeu
	
	<<<<<<<<<<<<<<<<<<<<<<<<<<<<<<<<<<<<<<<<<
	
	\subsection{Structure de l'interface}
	
		\subsubsection{Le JLayeredPane}
		
		
		L'interface de cette application a été fondée sur un système de couches de textures. Il a donc été nécessaire de rechercher quel objet avait la possibilité de contenir une multitude de \verb{JPanel{ et de les afficher selon l'ordre que le dévelopeur aurait choisit. Il s'agit de la classe \verb{JLayeredPane{.
		Nous avons donc dans un premier temps, implémenté une interface graphique basique sur laquel nous avont défini un \verb{JlayeredPane{ comme \verb{ContentPane{ de la fenêtre. Ainsi, nous pouvons désormais contenir l'ensemble des couches de textures, qui seront nécessaire à la construction et au bon fonctionnement de cette interface.
		
		\subsubsection{Les JPanel(s)
		
		Les \verb{JPanel{ qui constituent cette interface sont principalement au nombre de cinq :
		\begin{itemize}
		\item \verb{connection{
		\item \verb{formulaire{ 
		\item \verb{tacticPanel{ 
		\item \verb{texturePanel{
		\item \verb{playerTexture{
		\end{itemize}
		
			\paragraph{Page de connexion}
		
				Le \verb{JPanel{ \verb{Connexion{ ne contient que l'image de fond visible lors du démarrage de l'application. C'est aussi sur cette image que l'on a dessiné le cadre contenant le formulaire de connexion.
				\verb{formulaire{ est un \verb{Jpanel{ dont le \verb{Layout{ a été instancié à l'aide d'un \verb{GridLayout{ pour la mise en forme du formulaire. Ce \verb{JPanel{ a été redimensioné de manière à faire coincider parfaitement avec le cadre dessiné sur l'image de connexion. De plus, \verb{formulaire{ s'est vu attribuer un statut de transparence pour un rendu homogène de la page de connexion.
				\verb{JLayeredPane{ permet d'ajouter des \verb{JPanel{ les un au-dessus des autres selon un indexage précis allant de 1 à -n, nous avons donc superposé ces deux \verb{JPanel{ le plus "au dessus" possible de manière à placer la page de connexion au premier plan de l'application.
		
			\paragraph{Textures de Map}
			
				L'affichage de la \verb{Map{ sera confié à un \verb{JPanel{. Celui-ci aura pour but de fournir un affichage complet des textures. Nous verrons par la suite comment nous remplirons ce \verb{JPanel{ d'affichage, mais sachez qu'il est définit en tant que \verb{GridLayout{ au nombre de case en largeur et en hauteur de la map.
				Chaques cellules de ce \verb{JPanel{ sera remplis avec un jeu d'objet dérivant de la classe \verb{JPanel{ adapté à chaques texture. On retrouve pour les textures les classes suivantes :
				\begin{itemize}
				\item \verb{GrassTesture{ : représentation de l'herbe
				\item \verb{RockTexture{ : représentation des rochers
				\item \verb{WaterTexture{ : représentation de l'eau
				\item \verb{TreeTexture{ : représentation des arbres			
				\end{itemize}
			
		\subsubsection{Texture de Joueurs}
			
			En ce qui concerne les textures des personnages du jeu, ont en compte deux. La première sera la représentation de notre personnage. La seconde sera celle appliqué à tous les autres personnages présents sur la map. On notera que le personnage est représenté par un jeux de quatre images de texture. Une pour chaques directions dans laquel le personnage est censé se déplacer.
			La particularité des textures de joueurs est qu'il ne s'agit pas de \verb{JPanel{. Ici nous avons utilisé des objets de type \verb{Image{ que nous dessinerons sur le contexte graphique graĉe à la méthode \verb{drawImage(){ au coordonnées du personnage.
				
	\subsection{Fonctionnement de l'interface}
				
		\subsubsection{L'initialisation de l'interface graphique}
			
			\paragraph{La construction de la map}
			
				L'interface graphique avant même d'apparaître doit être apte à offrir à l'utilisateur toutes les informations. On va dons dans un premier temps instancier un objet de type \verb{Manager{ dans le but de récupérer la grille afin de l'interpreter et de l'afficher. On récupère donc ses dimensions. Il se pose alors un problème. L'idée est de générer une fenêtre de taille fixe mais pour une grille d'une taille infinie. Il faut donc trouver la taille en pixel des cellules de la grille sachant qu'elles doivent être carré, car représentée par des images de textures de 64 par 64 pixels. Pour cela, nous avons alors implémentés un algorithme qui effectue les actions suivantes : On va dans un premier temps savoir si la largeur estimé est plus grande que celle de la fenetre. Si ce n'est pas le cas, on regarde en fonction de la hauteur. Si la hauteur estimé dépasse celle de la fenetre, on recalcule le coté de la case en fonction de la hauteur. Sinon les images dans leurs tailles orginiales rentrerons dans la fenêtre. Sans trop détailler, nous allons discrétiser les cas possible et calculer le coté d'une case en fonction soit de la hauteur soit de la largeur.
				L'étape suivante sera de calculer la position de la grille dans la fenêtre afin de la centrer.
				Une fois le coté de la case déterminé, on peut interpréter la grille et affecter une texture a chaques cellules lues. Pour cela on fera appels au méthodes \verb{add...Texture(){.
				On ajoute alors le panel de la map au \verb{ContentPane{ de la fenêtre.
				
			\paragraph{La page de connexion}
			
				La page de connexion se base sur deux \verb{JPanel{ un pour afficher l'image de fond qui dissimulera la map texturé déjà présente et un autre pour le formulaire de connxexion.
				Pour l'image de fond nous ferons appel a un objet dérivé de \verb{JPanel{ qui sera \verb{ConnexionTexture{. Ne nous reste alors qu'a créer un \verb{JPanel{ formulaire de layout \verb{GridLayout{ pour contenir les labels les champs de saisies et le bouton de validation.
				
			\paragraph{La méthode \verb{update(){}
				
					La méthode \verb{update(){ est sans doute un élément nécessaire au bon fonctionnement de cette interface. Cette méthode lit les paquets reçu par le client, et offre une interprétation pour chacun de ces paquets. Ainsi, nous retrouvons les actions à effectuer, qu'il s'agisse d'une connexion de notre part ou non, d'un déplacement ou encore d'une déconnexion. On notera aussi que le refus de connexion par le serveur a aussi été traité par l'affichage d'une boîte de dialogue dont la particularité est qu'elle ne dévoile pas le contenu de l'erreur. 
		
		\subsubsection{Connexion, déconnexion et déplacements}
		
			\paragraph{La connexion}
				
				 Le mécanisme de connexion passe d'abord par l'action effectuée lors de l'appui sur le bouton de connexion. On va donc vérifier l'intégrité des différents champs du formulaire de connexion ainsi que l'accessibilité du serveur. Une fois la requête envoyée, la fonction \verb{update(){ interprète la réponse du serveur. Les panels de connexion sont alors supprimé et laissent place à la map. Les joueurs eux sont ajouté en parallèle puisque le serveur a envoyé en réalité deux message.
				 On appelle donc la méthode \verb{displayPlayers(){ qui se chargera de lire la hashmap des joueurs connectés et de les dessiner sur un \verb{JPanel{ placé sur une couche du \verb{JLayeredPane{. On appelle alors une méthode pour ajouter le bouton de déconnexion.
				 Lors de la connexion d'un nouveau joueur lors de notre présence sur le jeu, un traitement spécifique a été mis en place dans la méthode \verb{update(){ qui va rafraîchir le panel \verb{PlayerTexture{. Ceci aura pour effet de dessiner un joueur en plus à ses coordonnées.

			\paragraph{Les déplacements}
				
				Les déplacement se décomposent en deux cas, le premier est la demande de déplacement, la seconde est l'autorisation. La demande s'effectue à l'aide des \verb{Listener{ mis en place sur les flèches du clavier. Une requête est alors envoyée au serveur via le client.
				Dans le cas d'un déplacement validé par le serveur la réponse est traitée par la mathode \verb{update(){. On transmet alors au \verb{PlayerTexture{ les diverses information nécessaire à la mise à jour de ce panel. Ainsi, on va dans un premier temps détecter s'il s'agit de notre personnage ou celui d'un autre dans le but d'attribuer la texture associée. On redessine alors une texture de \verb{Grass{ à la position de l'ancienne coordonnée et la texture asssociée à la direction à la nouvelle coordonnée.
 				
			\paragraph{La déconnexion}

				Lors de la déconnexion d'un joueur, un rafraîchissement de la couche joueur est appellée ce qui entraîne l'application de la texture d'herbe sur ses coordonnées. En revanche lors d'une déconnexion intempestive ou lors de l'appuis sur le bouton mis à disposition, on fait appel à la méthode \verb{disconnection(){. Celle-ci fermera le client et l'envoie de données. Le bouton de déconnexion est alors retiré et on réinstancié les panels de la page de connexion.

\section{Couche réseau}
    
      Il n'y a qu'une seule classe utilisée pour la couche réseau, il s'agit de la classe Client.
      \newline
      
      Elle a comme attributs :
      \newline
      
      \begin{itemize}
    
	\item Les constantes identifiant les différents types de paquet
	\item L'en tête des paquets
	\item Le port et l'adresse IP du serveur, qui sont lues à partir d'un fichier de configuration
	\item Un Socket relié au serveur
	\item BufferedInputStream et BufferedOutputStream pour pouvoir dialoguer avec le serveur (en flux, par TCP)
	\item Une référence vers le Manager, classe principale du moteur
	\item Un Timer qui, une fois la connexion avec le serveur effectuée, envoie des requêtes de type "Ping" toutes 
	      les secondes pour vérifier si la connexion entre le serveur et le client existe toujours\newline
	
      \end{itemize}
	
      Lors de l'initialisation, si la connexion entre le serveur et le client s'établit, un nouveau Thread d'écoute est lancé.
      Son seul rôle est de lire en permanence le flux d'entrée, et si des données sont reçues, vérifier si le champ d'en tête est 
      valide. Si ce dernier est valide, le Thread extrait les données de la trame et les envoie au Manager pour qu'il traite 
      les informations.
      \newline
      
      La classe dispose également d'une méthode permettant de fermer la connexion entre le serveur et le client, 
      en fermant les flux d'entrée et de sortie, puis en fermant le Socket.
      \newline
      
      Enfin, la méthode sendData() permet d'envoyer les données vers le serveur. Elle reçoit les données à envoyer sous forme 
      de tableau de Byte ainsi que la longueur du paquet à envoyer, et envoie le paquet au serveur.
      \newline
      
      Evidemment, toutes les exceptions sont gérées et des logs sont générés pour chaque action.
    
    \section{Moteur}
    
      Le fichier GameClient.properties contient le port et l'adresse du serveur, log4j.properties contient la configuration des logs
      et standard\_map.txt contient la représentation textuelle de la carte.
      \newline
      
      La seule bibliothèque externe utilisée dans ce projet est log4j (génération de fichiers de log).
      \newline
      
      Il existe 2 énumérations, une représentant les types de décor (SetType), et l'autre représentant les directions 
      de déplacement possibles (Direction).
      \newline
      
      La surface de jeu est représentée par l'instanciation de la classe Map. Celle-ci dispose de plusieurs attributs :
      \newline
      
      \begin{itemize}
    
	\item La taille maximale de la carte (en hauteur et en largeur, fixées respectivement à 75 et 100)
	\item La taille effective de la carte
	\item Une HashMap de joueurs connectés, indexés par leur index
	\item Une référence vers le joueur local
	\item Un tableau à 2 dimensions d'objets de type Cell, représentant une case de la carte
	\newline
	
      \end{itemize}
      
      Lors de l'initialisation de la carte, le fichier "standard\_map.txt" est lu et interprété, ce qui remplit le tableau 
      représentant la carte. Le fichier contenant la carte doit évidemment être bien formé, ce qui est vérifié grâce à 
      l'utilisation de la méthode match() et une expression régulière décrivant le fichier tel qu'il devrait être.
      \newline
      
      La surface de jeu étant représentée par une grille, une classe Coordinate a été utilisée afin de pouvoir désigner 
      chacune des cases de la grille, et de pouvoir les comparer entre elles ou faire des opérations 
      (translation dans une direction, position relative de 2 cases).
      \newline
      
      Les classes Player et Set, qui représentent respectivement un joueur et du décor, étendent toutes les deux la classe 
      abstraite Cell, qui ne dispose que d'une méthode renvoyant le type d'élément de l'objet 
      (0 si c'est un joueur, 1 si c'est un décor, 2 si la case est libre).
      \newline
      
      Il existe 3 types de décor "obstacles", avec les mêmes propriétés (seul l'aspect visuel change) :
      \newline
      
      \begin{itemize}
    
	\item Les rochers
	\item Les arbres
	\item L'eau
	\newline
	
      \end{itemize}

      Chacun de ces décors représente un obstacle pour les joueurs. Ils ne peuvent pas les traverser mais ces décors ne
      gênent toutefois pas la vision.
      \newline
      
      Les joueurs présents sur la grille sont des instanciations de la classe Player, qui ne dispose que de 3 attributs :
      \newline
      
      \begin{itemize}
    
	\item Le pseudo du joueur
	\item L'index du joueur (son identifiant unique)
	\item Ses coordonnées
	\newline
	
      \end{itemize}
      
      Le but du projet étant de mettre en pratique les différents protocoles réseaux, et non de réaliser un jeu, 
      nous n'avons pas jugé utile de développer le côté "RPG" de l'application. C'est pourquoi un joueur ne dispose pas 
      de points de vie, de points de mana ou autres attributs relatifs à un personnage de RPG.
      \newline
      
      Enfin, c'est dans la classe Manager que se passent tous les traitements, que ce soit les données reçues du serveur 
      et extraites par la classe Client, ou les différentes actions du joueur local.
      \newline
      
      A la création du Manager, la construction de la Map est lancée.
      \newline
      
      Lors de l'initialisation, qui correspond au moment où l'utilisateur essaye de se connecter, un objet Client est 
      instancié et initialisé.
      \newline
      
      C'est dans cette classe que les paquets extraits par le Thread d'écoute de la classe Client sont traités dans la méthode 
      paquetOpeningFactory(). En fonction du type du paquet, des traitements seront effectués. Par exemple, si le type du paquet
      est 1, c'est une réponse positive du serveur à la demande de connexion de l'utilisateur. Si le type du paquet est 34, cela 
      veut dire qu'un joueur s'est déplacé, et le paquet contient l'index du joueur qui a bougé ainsi que son ancienne et 
      nouvelle coordonnée.
      \newline
      
      Les méthodes "dataObserver..." permettent ensuite de communiquer à l'IHM, via la méthode setData(), les données traitées,
      grâce à un design pattern Observable/Observer.
      \newline
      
      Le Manager dispose de 3 méthodes :
      \newline
      
      \begin{itemize}
    
	\item sendDataLogin() qui envoie les identifiants de l'utilisateur au serveur dans une requête de connexion
	\item sendDataMove() qui envoie une requête de déplacement dans la direction fournie au serveur
	\item sendDataLogout() qui indique au serveur que l'utilisateur s'est déconnecté
	\newline
	
      \end{itemize}
      
      Cette méthode est appelée par la méthode logout(), qui fait ensuite appel à la méthode closeClient() du Client 
      pour fermer les flux et le Socket en plus de prévenir le serveur de la déconnexion.
      Chacune de ces méthodes fait appel à la méthode sendData() du Client, en lui fournissant un tableau de données 
      et la longueur du champ de données.
      
  \chapter{Serveur}
  
    \section{Moteur}
    
      Le moteur travail principalement avec une file des joueurs en attente d'être placés sur la carte, une file de joueurs devant être 
      éjectés de la carte (perte de connexion, a simplement quitté le jeu), une carte 'du monde' (chargée depuis un fichier texte) et des joueurs.
      \newline
      
      La carte du monde est représentée par une structure WorldMap ayant :
      \newline
      
      \begin{itemize}
  
	\item Une longueur et une largeur
	\item Un tableau à deux dimensions de cellules (Cell)
	\item Une structure de type PlayerManagement\newline
	
      \end{itemize}
      
      La structure PlayerManagement est une structure de gestion des joueurs, elle permet de savoir combien de joueurs sont actuellement sur la carte
      et maintient un tableau de ces derniers. Cette structure est composée des éléments suivants :
      \newline
    
      \begin{itemize}
  
	\item Un tableau de joueurs
	\item Une pile des index du tableau qui sont libres et peuvent être utilisés
	\item Un entier représentant le nombre de joueurs présents sur la carte\newline
	
      \end{itemize}
    
      Une cellule (Cell) est une union qui représentera soit un décor, soit une case avec un joueur :
      \newline
    
      \begin{itemize}
  
	\item Un entier représentant le type de l'union
	\item Une case de décor (SetCell)
	\item Une case joueur (Player Cell)\newline
	
      \end{itemize}
      
      Une case de décor (SetCell) contient une énumeration pour définir le décor. Une case joueur (Player Cell) contient un structure de type Player.
      \newline
      
      Enfin, une structure joueur (Player) est composée des élements suivants :
      \newline

      \begin{itemize}
  
	\item Un entier représentant l'index de l'objet Player dans le tableau de PlayerManagement
	\item Une structure pour conserver la position du joueur sur la matrice carte (Position)
	\item Une chaine de caractères pour le login
	\item Une structure regroupant ce qui concerne la couche réseau (Connection)\newline
	
      \end{itemize}
      
    \section{Couche réseau}
  
  \part{Test}
  
      \chapter{Presentation du protocole de test}
	
	Afin d'eprouver la solidite du serveur nous avons modifie le programme client. Une fois l'etape d'authentification
	passee, le client rentre dans une boucle infinie de deplacements aleatoires sur la carte.
	\newline
	
	Ce protocol de test a ete mise en place et effectue sur les postes de l'universite dans la salle B532. Les postes
	possedent les caracteristiques suivantes :

	\begin{itemize}
    
	  \item Processeur
	  \item RAM
	  \item Carte reseau
	  \item OS\newline
	  
	\end{itemize}

      \chapter{Analyse des resultats}
	
	Sur le poste executant le serveur, la bande passante reseau consomee ne depasse jamais les 300 ko/s quel que soit 
	le nombre de clients.

  \part{Analyse critique}

    \chapter{Points forts}
    
    \chapter{Points faibles}
    
\part{Conclusion}

  \chapter{Difficultés rencontrées}
  
  \chapter{Ce que nous avons appris}
  
\part{Annexe}

\end{document}
